\documentclass[a4paper,10pt]{article}
\usepackage[utf8]{inputenc}
\usepackage{amsmath, amsthm, amssymb, amsfonts}
\usepackage{slashed}
\usepackage{mathtools}

%opening
\title{Final Project - Radiation of Gluon Jets}
\author{Georgios Billis}

\begin{document}

\maketitle

\section{}
For this project I ignore the masses of quarks and electron whereas I assume that gluons do have a 
 small non-zero mass $\mu$ and in the end of calculation it will be taken to zero.\\
 \indent The diagram cotributing to $e^{-}e^{+}\rightarrow q\bar{q}$ is at the end of this report (to save time I 
 did it by hand). This is the only one contributing since we do take into consideration only the one particle
 irreducible diagrams (in contrast to the diagrams that a gluon-quark loop appears at each of the final state
 external legs). For this diagram the matrix element is given below:\\
 \begin{align*}
  i\mathcal{M}_{g.loop} &=Q_{f}\bar{u}(p') \Big[\int \frac{d^{4}k}{(2\pi)^{4}}(-ig\gamma_{\rho}T^{\alpha})\frac{i}{\slashed{k}'+i\epsilon}(-ie\gamma_{\nu})\frac{i}{\slashed{k}+i\epsilon}(-ig\gamma_{\sigma}T^{\beta})\frac{-ig^{\rho\sigma}\delta^{\alpha\beta}}{(p-k)^{2}-{\mu}^{2}+i\epsilon}\Big]\upsilon(p)\times \\
               \times &\frac{-ig^{\mu\nu}}{q^{2}}\bar{\upsilon}(q_{2})(-ie\gamma_{\mu})u(q_{1})=\\
               &=-ig^{2}Q_{f}tr(T^{\alpha}T^{\alpha})\bar{u}(p')\Big[\int \frac{d^{4}k}{(2\pi)^{4}}\gamma_{\rho}\frac{1}{\slashed{k}'+i\epsilon}\gamma^{\nu}\frac{1}{\slashed{k}+i\epsilon}\gamma^{\rho}\frac{1}{(p-k)^{2}-{\mu}^{2}+i\epsilon}\Big]\upsilon(p)\times \\
               &\times \bar{\upsilon}(q_{2})\Big[(-ie)^{2}\gamma_{\nu}\frac{-i}{q^{2}}\Big]u(q_{1})=\\
               &=-ig^{2}\bar{u}(p')\Big[\int\frac{d^{4}k}{(2\pi)^{4}} \frac{\gamma_{\rho}\slashed{k}'\gamma^{\nu}\slashed{k}\gamma^{\rho}}{(k'^{2}+i\epsilon)(k^{2}+i\epsilon)((p-k)^{2}-\mu^{2}+i\epsilon)}\Big]\upsilon(p)\times i\mathcal{M}_{lep}_{\nu}
 \end{align*}
 where 
 \begin{align*}
i\mathcal{M}_{\nu}_{lep}=\bar{\upsilon}(q_{2})\Big[(-ie)^{2}Q_{f}\gamma_{\nu}\frac{-i}{q^{2}}\Big]u(q_{1}) 
 \end{align*}

and the one loop correction to the quark-gluon vertex is given by the other part, i.e.:
 
 \begin{align*}
  \bar{u}(p')\delta\Gamma^{\nu}\upsilon(p)=-ig^{2}\bar{u}(p')\Big[\int\frac{d^{4}k}{(2\pi)^{4}} \frac{\gamma_{\rho}\slashed{k}'\gamma^{\nu}\slashed{k}\gamma^{\rho}}{(k'^{2}+i\epsilon)(k^{2}+i\epsilon)((p-k)^{2}-\mu^{2}+i\epsilon)}\Big]\upsilon(p)
 \end{align*}

 with $\Gamma^{\nu}=\gamma^{\nu}+\delta\Gamma^{\nu}$\\
 
 In order to work out the loop momentum integral the method of Feynman parameters has to be introduced which simply 
 combines the three propagators into a single quadratic polynomial in k, (for our case) raised in the third power.\\
Working only the denominator part we have:

\begin{align*}
&\frac{1}{(k'^{2}+i\epsilon)(k^{2}+i\epsilon)((p-k)^{2}-\mu^{2}+i\epsilon)}=\\
& =\int dx dy dz \delta(x+y+z-1)\frac{2!}{[y((p-k)^{2}-\mu^{2}+i\epsilon)+x(k^{2}+i\epsilon)+z(k'^{2}+i\epsilon)]^{3}}
\end{align*}

Having in mind that $k'=k+q$ and that $x+y+z=1$ we have that 

\begin{align*}
 &\int dx dy dz \delta(x+y+z-1)\frac{2!}{[y(p-k)^{2}-y\mu^{2}+xk^{2}+z(q-k)^{2}+i\epsilon]^{3}}
\end{align*}
 
 and having in mind that quarks are considered to be massless i.e. $p'^{2}=p^{2}=0$, I obtain:
 
\begin{align*}
 &\int dxdydz\delta(x+y+z-1)\frac{2}{[k^{2}-2k\cdot(yp-zq)-y\mu^{2}+zq^{2}+i\epsilon]^{3}}=\\
 &=\int dxdydz\delta(x+y+z-1)\frac{2}{[(k-(yp-zq))^{2}-(yp-zq)^{2}-y\mu^{2}+zq^{2}+i\epsilon]^{3}}=\\
 &=\int dxdydz\delta(x+y+z-1)\frac{2}{[l^{2}+zq^{2}(1-z)+2yzp\cdot q-y\mu^{2}+i\epsilon]^{3}}=\\
 &=\int dxdydz\delta(x+y+z-1)\frac{2}{[l^{2}+zq^{2}(1-z)+yz(p+q)^{2}-yzq^{2}-y\mu^{2}+i\epsilon]^{3}}=\\
  &=\int dxdydz\delta(x+y+z-1)\frac{2}{[l^{2}+zq^{2}(1-z)+yzp'^{2}-yzq^{2}-y\mu^{2}+i\epsilon]^{3}}=\\
   &=\int dxdydz\delta(x+y+z-1)\frac{2}{[l^{2}+zxq^{2}-y\mu^{2}+i\epsilon]^{3}}
\end{align*}

thus the denominator can be written as $D=l^{2}-\Delta+i\epsilon$ with $\Delta=-zxq^{2}+y\mu^{2}$. \\
Now the numerator:

\begin{align*}
 N^{\nu}&=\bar{u}(p')\Big[\gamma_{\rho}\slashed{k}'\gamma^{\nu}\slashed{k}\gamma^{\rho}\Big]\upsilon(p)=\\
 &=(-2)\bar{u}(p')\Big[\slashed{k}\gamma^{\nu}\slashed{k}'\Big]\upsilon(p)=(-2)\bar{u}(p')\Big[\slashed{k}\gamma^{\nu}(\slashed{k}+\slashed{q})\Big]\upsilon(p)
\end{align*}

but $k=l-zq-yp$,  leading to:

\begin{align*}
 N^{\nu}&=(-2)\bar{u}(p')\Big[(\slashed{l}-z\slashed{q}+y\slashed{p})\gamma^{\nu}(\slashed{q}+\slashed{l}-z\slashed{q}+y\slashed{p})\Big]\upsilon(p)
\end{align*}

 once again the fact that quarks are massless implies that: $\bar{u}(p')\slashed{p}'=0$ and $\slashed{p}u(p)=0$. Now the goal is to bring it 
 into a form similar to the Gordon Identity, so:
 
 \begin{align*}
   N^{\nu}&=(-2)\bar{u}(p')\Big[(\slashed{l}\gamma^{\nu}\slashed{l}+(y\slashed{p}-z\slashed{q})\gamma^{\nu}((1-z)\slashed{q}+y\slased{p})\Big]\upsilon(p)=\\
          &=(-2)\bar{u}(p')\Big[\frac{-l^{2}}{2}\gamma^{\nu}+y\slashed{p}\gamma_{\nu}(1-z)\slashed{q}-z\slashed{q}\gamma^{\nu}(1-z)\slashed{q}\Big]\upsilon(p)=\\
	  &=(-2)\bar{u}(p')\Big[\frac{-l^{2}}{2}\gamma^{\nu}-y(\slashed{p}'-\slashed{p})\gamma_{\nu}(1-z)\slashed{q}-z\slashed{q}\gamma^{\nu}(1-z)\slashed{q})\Big]\upsilon(p)=\\
	  &=(-2)\bar{u}(p')\Big[\frac{-l^{2}}{2}\gamma^{\nu}-(y+z)(1-z)\slashed{q}\gamma^{\nu}\slashed{q}\Big]\upsilon(p)=\\
	  &=(2)\bar{u}(p')\Big[\frac{l^{2}}{2}\gamma^{\nu}-(y+z)(1-z)q^{2}\gamma^{\nu}\Big]\upsilon(p)
\end{align*}

where we have used properties of the gamma matrices such as $\{\gamma^{\mu},\gamma^{\nu}\}=2g^{\mu\nu}$ and that 
$\slashed{a}\slashed{a}=a^{2}$.\\
The Gordon identity enables us to write the vertex in terms of the form factors $F_{1}(q^{2})$ and $F_{2}(q^{2})$ 
but as it can be seen here, $\delta\Gamma^{\nu}$ is only dependent in $\gamma^{\nu}$ concluding that I already have in 
hand the form factor $\delta F_{1}(q^{2})$. Concretely: 

\begin{align*}
 \bar{u}(p')\delta\Gamma^{\nu}\upsilon(p)=(-4ig^{2})\bar{u}(p')\Big[\int\frac{d^{4}k}{(2\pi)^{4}}\int dxdydz \delta(x+y+z-1)\frac{\frac{l^{2}}{2}-(y+z)(1-z)q^{2}}{D^{3}}\gamma^{\nu}\Big]\upsilon(p)
\end{align*}

 Now in order to perform the momentum integral we are going do a Wick rotation. This consists of a counter-clockwise
 rotation in the $l^{0}$ plane such that we avoid the poles that appear. The Euclidian variables are defined as:
 $l^{0}=il^{0}_{E}$ and $\vec{l}_{E}=\vec{l}$. After the rotation, the $k$ integral will be performed in the four
 dimensional sperical coordinates. I will avoid restating the formulas of the integrals but I will mention the ones
 that apply in our case:
 
 \begin{align*}
  \int \frac{d^{4}l}{(2\pi)^{4}}\frac{1}{[l^{2}-\Delta]^{n}}=\frac{i(-1)^{n}}{(4\pi)^{2}}\frac{1}{(n-1)(n-2)}\frac{1}{\Delta^{n-2}} 
 \end{align*}

 and

  \begin{align*}
  \int \frac{d^{4}l}{(2\pi)^{4}}\frac{l^{2}}{[l^{2}-\Delta]^{n}}=\frac{i(-1)^{n-1}}{(4\pi)^{2}}\frac{1}{(n-1)(n-2)(n-3)}\frac{1}{\Delta^{n-3}} 
 \end{align*}
 
 It can be seen that the second integral for this case $(n=3)$ is divergent. In order to address this issue, the 
 Pauli-Villars regularization will be employed. This consists in introducing a ficticious, heavy, particle whose 
 contribution is subtracted from that of the gluon. This results in a substitution in the gluon propagator:
 
 \begin{align*}
  \frac{1}{(p-k)^{2}-\mu^{2}+i\epsilon}\rightarrow \frac{1}{(p-k)^{2}-\mu^{2}+i\epsilon}-\frac{1}{(p-k)^{2}-\Lambda^{2}+i\epsilon}
 \end{align*}

Thus:

\begin{align*}
 \int \frac{d^{4}l}{(2\pi)^{4}}\Bigg(\frac{l^{2}}{[l^{2}-\Delta]^{3}}-\frac{l^{2}}{[l^{2}-\Delta_{\Lambda}]^{3}}\Bigg)=\frac{i}{(4\pi)^{2}}log\Big(\frac{\Delta_{\Lambda}}{\Delta}\Big)
 \end{align*}

 where $\Delta_{\Lambda}=-zxq^{2}+y\Lambda^{2}$, and for $\Lambda>>$ we have

 \begin{align*}
  \approx \frac{i}{(4\pi)^{2}}log\Big(\frac{y\Lambda^{2}}{\Delta}\Big)
 \end{align*}

 As indicated by the exercise, the renormalisation will be realised by subtraction at $q^{2}=0$ where the following 
 substitution will be made: \\ $\delta F_{1}(q^{2})\rightarrow \delta F_{1}(q^{2})-\delta F_{1}(0)$ such that the 
 condition $F_{1}(q^{2})=1$ is fullfilled. For the integral that is divergent the subtraction yields the following 
 term:
 
 \begin{align*}
 \frac{i}{(4\pi)^{2}}\Bigg[log\Big(\frac{y\Lambda^{2}}{\Delta}\Big)-log\Big(\frac{y\Lambda^{2}}{y\mu^{2}}\Big)\Bigg]=\frac{i}{(4\pi)^{2}}log\Big(\frac{y\mu^{2}}{\Delta}\Big) 
 \end{align*}

 and the non divergent integral is:
 
 \begin{align*}
  \int \frac{d^{4}l}{(2\pi)^{4}}\frac{1}{[l^{2}-\Delta]^{3}}=\frac{-i}{(4\pi)^{2}}\frac{1}{2\Delta}
 \end{align*}

 To sum up:
 
 \begin{align*}
  i\mathcal{M}_{g.loop} &=\frac{2g^{2}}{(4\pi)^{2}}\bar{u}(p')\Bigg[\int dxdydz \delta(x+y+z-1)\bigg(log\Big(\frac{y\mu^{2}}{\Delta}\Big)+\frac{(1-x)(1-z)q^{2}}{\Delta}\bigg)\gamma^{\nu}\Bigg]\upsilon(p) i\mathcal{M}_{\nu lep}=\\
			&=\frac{\alpha_{g}}{(2\pi)}\bar{u}(p')\Bigg[\int dxdydz \delta(x+y+z-1)\bigg(log\Big(\frac{y\mu^{2}}{\Delta}\Big)+\frac{(1-x)(1-z)q^{2}}{\Delta}\bigg)\gamma^{\nu}\Bigg]\upsilon(p) i\mathcal{M}_{\nu lep}
 \end{align*}

 leading to the form factor that includes the zero order corrections:
 
 \begin{align*}
  F_{1}(q^{2})=\Bigg[Q_{f}+\frac{\alpha_{g}Q_{f}}{(2\pi)}\int dxdydz \delta(x+y+z-1)\bigg(log\Big(\frac{y\mu^{2}}{\Delta}\Big)+\frac{(1-x)(1-z)q^{2}}{\Delta}\bigg)\Bigg]
 \end{align*}

 satisfying $F_{1}(q^{2})=0$. Therefore the cross section for the production of $q\bar{q}$ can be written as:
 
 \begin{align*}
  \sigma (e^{-}e^{+}\rightarrow q\bar{q})= \frac{4\pi\alpha^{2}}{3s}\cdot 3\left| F_{1}(q^{2})\right| ^{2}
 \end{align*}

 with $F_{1}(q^{2})$ containing the first order quark gluon corrections. 
\end{document}





\documentclass[a4paper,10pt]{report}
\usepackage[utf8]{inputenc}
\usepackage{amsmath, amsthm, amssymb, amsfonts}
\usepackage[inline]{enumitem}

% Title Page
\title{Final Project - Radiation of Gluon Jets}
\author{Georgios Billis}


